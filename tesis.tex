% ----------------------------------------------------------------------
%
%                          Tesis.tex
%
%----------------------------------------------------------------------
%
% Este fichero contiene el "documento maestro" del documento. Lo �nico
% que hace es configurar el entorno LaTeX e incluir los ficheros .tex
% que contienen cada sección.
%
%----------------------------------------------------------------------
%
% Los ficheros necesarios para este documento son:
%
%       TeXiS/* : ficheros de la plantilla TeXiS.
%       Cascaras/* : ficheros con las partes del documento que no
%          son cap�tulos ni ap�ndices (portada, agradecimientos, etc.)
%       Capitulos/*.tex : cap�tulos de la tesis
%       Apendices/*.tex: ap�ndices de la tesis
%       constantes.tex: constantes LaTeX
%       config.tex : configuración de la "compilación" del documento
%       guionado.tex : palabras con guiones
%
% Para la bibliograf�a, adem�s, se necesitan:
%
%       *.bib : ficheros con la información de las referencias
%
% ---------------------------------------------------------------------

\documentclass[12pt,letterpaper,oneside]{book}

%
% Definimos  el   comando  \compilaCapitulo,  que   luego  se  utiliza
% (opcionalmente) en config.tex. Quedar�a  mejor si tambi�n se definiera
% en  ese fichero,  pero por  el modo  en el  que funciona  eso  no es
% posible. Puedes consultar la documentación de ese fichero para tener
% m�s  información. Definimos tambi�n  \compilaApendice, que  tiene el
% mismo  cometido, pero  que se  utiliza para  compilar  �nicamente un
% ap�ndice.
%
%
% Si  queremos   compilar  solo   una  parte  del   documento  podemos
% especificar mediante  \includeonly{...} qu� ficheros  son los �nicos
% que queremos  que se incluyan.  Esto  es �til por  ejemplo para s�lo
% compilar un cap�tulo.
%
% El problema es que todos aquellos  ficheros que NO est�n en la lista
% NO   se  incluir�n...  y   eso  tambi�n   afecta  a   ficheros  de
% la plantilla...
%
% Total,  que definimos  una constante  con los  ficheros  que siempre
% vamos a querer compilar  (aquellos relacionados con configuración) y
% luego definimos \compilaCapitulo.
\newcommand{\ficherosBasicosTeXiS}{%
TeXiS/TeXiS_pream,TeXiS/TeXiS_cab,TeXiS/TeXiS_bib,TeXiS/TeXiS_cover,%
TeXiS/TeXiS_part%
}
\newcommand{\ficherosBasicosTexto}{%
constantes,guionado,Cascaras/bibliografia,config%
}
\newcommand{\compilaCapitulo}[1]{%
\includeonly{\ficherosBasicosTeXiS,\ficherosBasicosTexto,Capitulos/#1}%
}

\newcommand{\compilaApendice}[1]{%
\includeonly{\ficherosBasicosTeXiS,\ficherosBasicosTexto,Apendices/#1}%
}

%- - - - - - - - - - - - - - - - - - - - - - - - - - - - - - - - - - -
%            Pre�mbulo del documento. Configuraciones varias
%- - - - - - - - - - - - - - - - - - - - - - - - - - - - - - - - - - -

% Define  el  tipo  de  compilación que  estamos  haciendo.   Contiene
% definiciones  de  constantes que  cambian  el  comportamiento de  la
% compilación. Debe incluirse antes del paquete TeXiS/TeXiS.sty
\include{config}

% Paquete de la plantilla
\usepackage{TeXiS/TeXiS}

% Incluimos el fichero con comandos de constantes
%---------------------------------------------------------------------
%
%                          constantes.tex
%
%---------------------------------------------------------------------
%
% Fichero que  declara nuevos comandos LaTeX  sencillos realizados por
% comodidad en la escritura de determinadas palabras
%
%---------------------------------------------------------------------

%%%%%%%%%%%%%%%%%%%%%%%%%%%%%%%%%%%%%%%%%%%%%%%%%%%%%%%%%%%%%%%%%%%%%%
% Comando: 
%
%       \titulo
%
% Resultado: 
%
% Escribe el título del documento.
%%%%%%%%%%%%%%%%%%%%%%%%%%%%%%%%%%%%%%%%%%%%%%%%%%%%%%%%%%%%%%%%%%%%%%
\def\titulo{%
  Marco de trabajo para predecir y mitigar posibles ca\'idas de servicio
 usando Machine Learning aplicado sobre los sistemas LMS para soportar
 alto tr\'afico de la Universidad Aut\'onoma Gabriel Ren\'e Moreno
}

%%%%%%%%%%%%%%%%%%%%%%%%%%%%%%%%%%%%%%%%%%%%%%%%%%%%%%%%%%%%%%%%%%%%%%
% Comando: 
%
%       \autor
%
% Resultado: 
%
% Escribe el autor del documento.
%%%%%%%%%%%%%%%%%%%%%%%%%%%%%%%%%%%%%%%%%%%%%%%%%%%%%%%%%%%%%%%%%%%%%%
\def\autor{Luis Ariel Vega Soliz}

% Variable local para emacs, para  que encuentre el fichero maestro de
% compilación y funcionen mejor algunas teclas rápidas de AucTeX

%%%
%%% Local Variables:
%%% mode: latex
%%% TeX-master: "tesis.tex"
%%% End:


% Sacamos en el log de la compilación el copyright
\typeout{Copyright Marco Antonio and Pedro Pablo Gomez Martin}

%
% "Metadatos" para el PDF
%
\ifpdf\hypersetup{%
    pdftitle = {\titulo},
    pdfsubject = {Plantilla de Tesis},
    pdfkeywords = {Plantilla, LaTeX, tesis, trabajo de
      investigación, trabajo de Master},
    pdfauthor = {\textcopyright\ \autor},
    pdfcreator = {\LaTeX\ con el paquete \flqq hyperref\frqq},
    pdfproducer = {pdfeTeX-0.\the\pdftexversion\pdftexrevision},
    }
    \pdfinfo{/CreationDate (\today)}
\fi


%- - - - - - - - - - - - - - - - - - - - - - - - - - - - - - - - - - -
%                        Documento
%- - - - - - - - - - - - - - - - - - - - - - - - - - - - - - - - - - -
\begin{document}

% Incluimos el  fichero de definición de guionado  de algunas palabras
% que LaTeX no ha dividido como deber�a
\input{guionado}

% Marcamos  el inicio  del  documento para  la  numeración de  p�ginas
% (usando n�meros romanos para esta primera fase).
\frontmatter

%---------------------------------------------------------------------
%
%                          configCover.tex
%
%---------------------------------------------------------------------
%
% cover.tex
% Copyright 2009 Marco Antonio Gomez-Martin, Pedro Pablo Gomez-Martin
%
% This file belongs to the TeXiS manual, a LaTeX template for writting
% Thesis and other documents. The complete last TeXiS package can
% be obtained from http://gaia.fdi.ucm.es/projects/texis/
%
% Although the TeXiS template itself is distributed under the 
% conditions of the LaTeX Project Public License
% (http://www.latex-project.org/lppl.txt), the manual content
% uses the CC-BY-SA license that stays that you are free:
%
%    - to share & to copy, distribute and transmit the work
%    - to remix and to adapt the work
%
% under the following conditions:
%
%    - Attribution: you must attribute the work in the manner
%      specified by the author or licensor (but not in any way that
%      suggests that they endorse you or your use of the work).
%    - Share Alike: if you alter, transform, or build upon this
%      work, you may distribute the resulting work only under the
%      same, similar or a compatible license.
%
% The complete license is available in
% http://creativecommons.org/licenses/by-sa/3.0/legalcode
%
%---------------------------------------------------------------------
%
% Fichero que contiene la configuración de la portada y de la 
% primera hoja del documento.
%
%---------------------------------------------------------------------


% Pueden configurarse todos los elementos del contenido de la portada
% utilizando comandos.

%%%%%%%%%%%%%%%%%%%%%%%%%%%%%%%%%%%%%%%%%%%%%%%%%%%%%%%%%%%%%%%%%%%%%%
% Institución/departamento asociado al documento.
% \institucion{Nombre}
% Puede tener varias l�neas. Se utiliza en las dos portadas.
% Si no se indica aparecer� vac�o.
%%%%%%%%%%%%%%%%%%%%%%%%%%%%%%%%%%%%%%%%%%%%%%%%%%%%%%%%%%%%%%%%%%%%%%
\institucion{%
UNIVERSIDAD AUT\'ONOMA ``GABRIEL REN\'E MORENO''\\
FACULTAD DE INGENIER\'IA EN CIENCIAS DE LA COMPUTACI\'ON Y TELECOMUNICACIONES\\
\vspace{36pt}
``UAGRM SCHOOL OF ENGINEERING''
}

%%%%%%%%%%%%%%%%%%%%%%%%%%%%%%%%%%%%%%%%%%%%%%%%%%%%%%%%%%%%%%%%%%%%%%
% Título del grado:
% \tituloGrado{titulo}
% Nota:
% Aquí se coloca el título del grado al que se va a optar.
%%%%%%%%%%%%%%%%%%%%%%%%%%%%%%%%%%%%%%%%%%%%%%%%%%%%%%%%%%%%%%%%%%%%%%
\tituloGrado{%
MAESTR\'IA EN DIRECCI\'ON ESTRAT\'EGICA EN INGENIER\'IA DE SOFTWARE
}

%%%%%%%%%%%%%%%%%%%%%%%%%%%%%%%%%%%%%%%%%%%%%%%%%%%%%%%%%%%%%%%%%%%%%%
% Título del documento:
% \tituloPortada{titulo}
% Nota:
% Si no se define se utiliza el del \titulo. Este comando permite
% cambiar el t�tulo de forma que se especifiquen d�nde se quieren
% los retornos de carro cuando se utilizan fuentes grandes.
%%%%%%%%%%%%%%%%%%%%%%%%%%%%%%%%%%%%%%%%%%%%%%%%%%%%%%%%%%%%%%%%%%%%%%
\tituloPortada{%
Marco de trabajo para predecir y mitigar posibles ca\'idas de servicio usando Machine Learning aplicado sobre los sistemas LMS para soportar alto tr\'afico de la Universidad Aut\'onoma Gabriel Ren\'e Moreno%
}

%%%%%%%%%%%%%%%%%%%%%%%%%%%%%%%%%%%%%%%%%%%%%%%%%%%%%%%%%%%%%%%%%%%%%%
% Subtítulo del documento:
% \subtituloPortada{subtitulo}
% Nota:
% Si no se define se utiliza el del \titulo. Este comando permite
% cambiar el t�tulo de forma que se especifiquen d�nde se quieren
% los retornos de carro cuando se utilizan fuentes grandes.
%%%%%%%%%%%%%%%%%%%%%%%%%%%%%%%%%%%%%%%%%%%%%%%%%%%%%%%%%%%%%%%%%%%%%%
\subtituloPortada{%
TRABAJO FINAL DE GRADO BAJO LA MODALIDAD DE TESIS PARA OPTAR AL T\'ITULO DE MAESTRO EN CIENCIAS %
}

%%%%%%%%%%%%%%%%%%%%%%%%%%%%%%%%%%%%%%%%%%%%%%%%%%%%%%%%%%%%%%%%%%%%%%
% Autor del documento:
% \autorPortada{Nombre}
% Se utiliza en la portada y en el valor por defecto del
% primer subt�tulo de la segunda portada.
%%%%%%%%%%%%%%%%%%%%%%%%%%%%%%%%%%%%%%%%%%%%%%%%%%%%%%%%%%%%%%%%%%%%%%
\autorPortada{Luis Ariel Vega Soliz}

%%%%%%%%%%%%%%%%%%%%%%%%%%%%%%%%%%%%%%%%%%%%%%%%%%%%%%%%%%%%%%%%%%%%%%
% Lugar de publicación:
% \lugarPublicacion{Lugar}
% Puede ser vacío. Aparece en la penúltima línea de ambas portadas
%%%%%%%%%%%%%%%%%%%%%%%%%%%%%%%%%%%%%%%%%%%%%%%%%%%%%%%%%%%%%%%%%%%%%%
\lugarPublicacion{Santa Cruz, Bolivia}

%%%%%%%%%%%%%%%%%%%%%%%%%%%%%%%%%%%%%%%%%%%%%%%%%%%%%%%%%%%%%%%%%%%%%%
% Fecha de publicación:
% \fechaPublicacion{Fecha}
% Puede ser vacío. Aparece en la última línea de ambas portadas
%%%%%%%%%%%%%%%%%%%%%%%%%%%%%%%%%%%%%%%%%%%%%%%%%%%%%%%%%%%%%%%%%%%%%%
\fechaPublicacion{Noviembre, 2023}

%%%%%%%%%%%%%%%%%%%%%%%%%%%%%%%%%%%%%%%%%%%%%%%%%%%%%%%%%%%%%%%%%%%%%%
% Imagen de la portada (y escala)
% \imagenPortada{Fichero}
% \escalaImagenPortada{Numero}
% Si no se especifica, se utiliza la imagen TODO.pdf
%%%%%%%%%%%%%%%%%%%%%%%%%%%%%%%%%%%%%%%%%%%%%%%%%%%%%%%%%%%%%%%%%%%%%%
\imagenPortada{Imagenes/Vectorial/UAGRMEscudoSOE}
\escalaImagenPortada{1.0}

%%%%%%%%%%%%%%%%%%%%%%%%%%%%%%%%%%%%%%%%%%%%%%%%%%%%%%%%%%%%%%%%%%%%%%
% Tipo de documento.
% \tipoDocumento{Tipo}
% Para el texto justo debajo del escudo.
% Si no se indica, se utiliza "TESIS DOCTORAL".
%%%%%%%%%%%%%%%%%%%%%%%%%%%%%%%%%%%%%%%%%%%%%%%%%%%%%%%%%%%%%%%%%%%%%%
\tipoDocumento{TESIS DE MAESTR\'IA}

%%%%%%%%%%%%%%%%%%%%%%%%%%%%%%%%%%%%%%%%%%%%%%%%%%%%%%%%%%%%%%%%%%%%%%
% Director del trabajo.
% \directorPortada{Nombre}
% Se utiliza para el valor por defecto del segundo subtítulo, donde
% se indica quién es el director del trabajo.
% Si se fuerza un subtítulo distinto, no hace falta definirlo.
%%%%%%%%%%%%%%%%%%%%%%%%%%%%%%%%%%%%%%%%%%%%%%%%%%%%%%%%%%%%%%%%%%%%%%
\directorPortada{Walterio Malatesta}

%%%%%%%%%%%%%%%%%%%%%%%%%%%%%%%%%%%%%%%%%%%%%%%%%%%%%%%%%%%%%%%%%%%%%%
% Texto del primer subt�tulo de la segunda portada.
% \textoPrimerSubtituloPortada{Texto}
% Para configurar el primer "texto libre" de la segunda portada.
% Si no se especifica se indica "Memoria que presenta para optar al
% t�tulo de Doctor en Inform�tica" seguido del \autorPortada.
%%%%%%%%%%%%%%%%%%%%%%%%%%%%%%%%%%%%%%%%%%%%%%%%%%%%%%%%%%%%%%%%%%%%%%
\textoPrimerSubtituloPortada{%
\textit{Informe t�cnico del departamento}  \\ [0.3em]
\textbf{Ingenier�a del Software e Inteligencia Artificial} \\ [0.3em]
\textbf{IT/2009/3}
}

%%%%%%%%%%%%%%%%%%%%%%%%%%%%%%%%%%%%%%%%%%%%%%%%%%%%%%%%%%%%%%%%%%%%%%
% Texto del segundo subt�tulo de la segunda portada.
% \textoSegundoSubtituloPortada{Texto}
% Para configurar el segundo "texto libre" de la segunda portada.
% Si no se especifica se indica "Dirigida por el Doctor" seguido
% del \directorPortada.
%%%%%%%%%%%%%%%%%%%%%%%%%%%%%%%%%%%%%%%%%%%%%%%%%%%%%%%%%%%%%%%%%%%%%%
\textoSegundoSubtituloPortada{%
\textit{Versi�n \texisVer}
}

%%%%%%%%%%%%%%%%%%%%%%%%%%%%%%%%%%%%%%%%%%%%%%%%%%%%%%%%%%%%%%%%%%%%%%
% \explicacionDobleCara
% Si se utiliza, se aclara que el documento est� preparado para la
% impresi�n a doble cara.
%%%%%%%%%%%%%%%%%%%%%%%%%%%%%%%%%%%%%%%%%%%%%%%%%%%%%%%%%%%%%%%%%%%%%%
\explicacionDobleCara

%%%%%%%%%%%%%%%%%%%%%%%%%%%%%%%%%%%%%%%%%%%%%%%%%%%%%%%%%%%%%%%%%%%%%%
% \isbn
% Si se utiliza, aparecer� el ISBN detr�s de la segunda portada.
%%%%%%%%%%%%%%%%%%%%%%%%%%%%%%%%%%%%%%%%%%%%%%%%%%%%%%%%%%%%%%%%%%%%%%
\isbn{978-84-692-7109-4}


%%%%%%%%%%%%%%%%%%%%%%%%%%%%%%%%%%%%%%%%%%%%%%%%%%%%%%%%%%%%%%%%%%%%%%
% \copyrightInfo
% Si se utiliza, aparecer� información de los derechos de copyright
% detr�s de la segunda portada.
%%%%%%%%%%%%%%%%%%%%%%%%%%%%%%%%%%%%%%%%%%%%%%%%%%%%%%%%%%%%%%%%%%%%%%
\copyrightInfo{\autor}


%%
%% Creamos las portadas
%%
\makeCover

% Variable local para emacs, para que encuentre el fichero
% maestro de compilación
%%%
%%% Local Variables:
%%% mode: latex
%%% TeX-master: "../Tesis.tex"
%%% End:


\include{Cascaras/dedicatoria}

%---------------------------------------------------------------------
%
%                      agradecimientos.tex
%
%---------------------------------------------------------------------
%
% agradecimientos.tex
% Copyright 2009 Marco Antonio Gomez-Martin, Pedro Pablo Gomez-Martin
%
% This file belongs to the TeXiS manual, a LaTeX template for writting
% Thesis and other documents. The complete last TeXiS package can
% be obtained from http://gaia.fdi.ucm.es/projects/texis/
%
% Although the TeXiS template itself is distributed under the 
% conditions of the LaTeX Project Public License
% (http://www.latex-project.org/lppl.txt), the manual content
% uses the CC-BY-SA license that stays that you are free:
%
%    - to share & to copy, distribute and transmit the work
%    - to remix and to adapt the work
%
% under the following conditions:
%
%    - Attribution: you must attribute the work in the manner
%      specified by the author or licensor (but not in any way that
%      suggests that they endorse you or your use of the work).
%    - Share Alike: if you alter, transform, or build upon this
%      work, you may distribute the resulting work only under the
%      same, similar or a compatible license.
%
% The complete license is available in
% http://creativecommons.org/licenses/by-sa/3.0/legalcode
%
%---------------------------------------------------------------------
%
% Contiene la p�gina de agradecimientos.
%
% Se crea como un cap�tulo sin numeraci�n.
%
%---------------------------------------------------------------------

\chapter{Agradecimientos}

\cabeceraEspecial{Agradecimientos}

\begin{FraseCelebre}
\begin{Frase}
A todos los que la presente vieren y entendieren.
\end{Frase}
\begin{Fuente}
Inicio de las Leyes Org\'anicas. Juan Carlos I
\end{Fuente}
\end{FraseCelebre}

Groucho Marx dec\'ia que encontraba a la televisi\'on muy educativa porque
cada vez que alguien la encend\'ia, \'el se iba a otra habitaci\'on a leer
un libro. Utilizando un esquema similar, nosotros queremos agradecer
al Word de Microsoft el habernos forzado a utilizar \LaTeX. Cualquiera
que haya intentado escribir un documento de m\'as de 150 p\'aginas con
esta aplicaci\'on entender\'a a qu\'e nos referimos. Y lo decimos porque
nuestra andadura con \LaTeX\ comenz\'o, precisamente, despu\'es de
escribir un documento de algo m\'as de 200 p\'aginas. Una vez terminado
decidimos que nunca m\'as pasar\'iamos por ah\'i. Y entonces ca\'imos en
\LaTeX.

Es muy posible que hubieramos llegado al mismo sitio de todas formas,
ya que en el mundo acad\'emico a la hora de escribir art\'iculos y
contribuciones a congresos lo m\'as extendido es \LaTeX. Sin embargo,
tambi\'en es cierto que cuando intentas escribir un documento grande
en \LaTeX\ por tu cuenta y riesgo sin un enlace del tipo ``\emph{Author
  instructions}'', se hace cuesta arriba, pues uno no sabe por donde
empezar.

Y ah\'i es donde debemos agradecer tanto a Pablo Gerv\'as como a Miguel
Palomino su ayuda. El primero nos ofreci\'o el c\'odigo fuente de una
programaci\'on docente que hab\'ia hecho unos a\~{n}os atr\'as y que nos sirvi\'o
de inspiraci\'on (por ejemplo, el fichero \texttt{guionado.tex} de
\texis\ tiene una estructura casi exacta a la suya e incluso puede
que el nombre sea el mismo). El segundo nos dej\'o husmear en el c\'odigo
fuente de su propia tesis donde, adem\'as de otras cosas m\'as
interesantes pero menos curiosas, descubrimos que a\'un hay gente que
escribe los acentos espa\~{n}oles con el \verb+\'{\i}+.

No podemos tampoco olvidar a los numerosos autores de los libros y
tutoriales de \LaTeX\ que no s\'olo permiten descargar esos manuales sin
coste adicional, sino que tambi\'en dejan disponible el c\'odigo fuente.
Estamos pensando en Tobias Oetiker, Hubert Partl, Irene Hyna y
Elisabeth Schlegl, autores del famoso ``The Not So Short Introduction
to \LaTeXe'' y en Tom\'as Bautista, autor de la traducci\'on al espa\~{n}ol. De
ellos es, entre otras muchas cosas, el entorno \texttt{example}
utilizado en algunos momentos en este manual.

Tambi\'en estamos en deuda con Joaqu\'in Ataz L\'opez, autor del libro
``Creaci�n de ficheros \LaTeX\ con {GNU} Emacs''. Gracias a �l dejamos
de lado a WinEdt y a Kile, los editores que por entonces utiliz�bamos
en entornos Windows y Linux respectivamente, y nos pasamos a emacs. El
tiempo de escritura que nos ahorramos por no mover las manos del
teclado para desplazar el cursor o por no tener que escribir
\verb+\emph+ una y otra vez se lo debemos a �l; nuestro ocio y vida
social se lo agradecen.

Por �ltimo, gracias a toda esa gente creadora de manuales, tutoriales,
documentaci�n de paquetes o respuestas en foros que hemos utilizado y
seguiremos utilizando en nuestro quehacer como usuarios de
\LaTeX. Sab�is un mont�n.

Y para terminar, a Donal Knuth, Leslie Lamport y todos los que hacen y
han hecho posible que hoy puedas estar leyendo estas l�neas.

\endinput
% Variable local para emacs, para  que encuentre el fichero maestro de
% compilaci�n y funcionen mejor algunas teclas r�pidas de AucTeX
%%%
%%% Local Variables:
%%% mode: latex
%%% TeX-master: "../Tesis.tex"
%%% End:


\include{Cascaras/resumen}

\ifx\generatoc\undefined
\else
%---------------------------------------------------------------------
%
%                          TeXiS_toc.tex
%
%---------------------------------------------------------------------
%
% TeXiS_toc.tex
% Copyright 2009 Marco Antonio Gomez-Martin, Pedro Pablo Gomez-Martin
%
% This file belongs to TeXiS, a LaTeX template for writting
% Thesis and other documents. The complete last TeXiS package can
% be obtained from http://gaia.fdi.ucm.es/projects/texis/
%
% This work may be distributed and/or modified under the
% conditions of the LaTeX Project Public License, either version 1.3
% of this license or (at your option) any later version.
% The latest version of this license is in
%   http://www.latex-project.org/lppl.txt
% and version 1.3 or later is part of all distributions of LaTeX
% version 2005/12/01 or later.
%
% This work has the LPPL maintenance status `maintained'.
% 
% The Current Maintainers of this work are Marco Antonio Gomez-Martin
% and Pedro Pablo Gomez-Martin
%
%---------------------------------------------------------------------
%
% Contiene  los  comandos  para  generar los  índices  del  documento,
% entendiendo por índices las tablas de contenidos.
%
% Genera  el  índice normal  ("tabla  de  contenidos"),  el índice  de
% figuras y el de tablas. También  crea "marcadores" en el caso de que
% se está compilando con pdflatex para que aparezcan en el PDF.
%
%---------------------------------------------------------------------


% Primero un poquito de configuración...


% Pedimos que inserte todos los epígrafes hasta el nivel \subsection en
% la tabla de contenidos.
\setcounter{tocdepth}{2} 

% Le  pedimos  que nos  numere  todos  los  epígrafes hasta  el  nivel
% \subsubsection en el cuerpo del documento.
\setcounter{secnumdepth}{3} 


% Creamos los diferentes índices.

% Lo primero un  poco de trabajo en los marcadores  del PDF. No quiero
% que  salga una  entrada  por cada  índice  a nivel  0...  si no  que
% aparezca un marcador "índices", que  tenga dentro los otros tipos de
% índices.  Total, que creamos el marcador "índices".
% Antes de  la creación  de los índices,  se añaden los  marcadores de
% nivel 1.

\ifpdf
   \pdfbookmark{\'Indices}{Indices}
\fi

% Tabla de contenidos.
%
% La  inclusión  de '\tableofcontents'  significa  que  en la  primera
% pasada  de  LaTeX  se  crea   un  fichero  con  extensión  .toc  con
% información sobre la tabla de contenidos (es conceptualmente similar
% al  .bbl de  BibTeX, creo).  En la  segunda ejecución  de  LaTeX ese
% documento se utiliza para  generar la verdadera página de contenidos
% usando la  información sobre los  capítulos y demás guardadas  en el
% .toc
\ifpdf
   \pdfbookmark[1]{\'Indice general}{Indice general}
\fi

\cabeceraEspecial{\'Indice general}
\renewcommand{\contentsname}{\'INDICE GENERAL}
\tableofcontents

\newpage 

% índice de cuadros
%
% La idea es semejante que para  el .toc del índice, pero ahora se usa
% extensión .lof (List Of Figures) con la información de las figuras.


\ifpdf
   \pdfbookmark[1]{\'Indice de cuadros}{indice de cuadros}
\fi

\cabeceraEspecial{\'Indice de cuadros}

\renewcommand{\listfigurename}{\'INDICE DE CUADROS}
\listoffigures

\newpage

% índice de tablas
% Como antes, pero ahora .lot (List Of Tables)

\ifpdf
   \pdfbookmark[1]{\'Indice de tablas}{indice de tablas}
\fi

\cabeceraEspecial{\'Indice de tablas}

\renewcommand{\listtablename}{\'INDICE DE TABLAS}
\listoftables

\newpage

% índice de figuras
%
% La idea es semejante que para  el .toc del índice, pero ahora se usa
% extensión .lof (List Of Figures) con la información de las figuras.


\ifpdf
   \pdfbookmark[1]{\'Indice de figuras}{indice de figuras}
\fi

\cabeceraEspecial{\'Indice de figuras}

\renewcommand{\listfigurename}{\'INDICE DE FIGURAS}
\listoffigures

\newpage

% índice de ilustraciones
%
% La idea es semejante que para  el .toc del índice, pero ahora se usa
% extensión .lof (List Of Figures) con la información de las figuras.


\ifpdf
   \pdfbookmark[1]{\'Indice de ilustraciones}{indice de ilustraciones}
\fi

\cabeceraEspecial{\'Indice de ilustraciones}

\renewcommand{\listfigurename}{\'INDICE DE ILUSTRACIONES}
\listoffigures

\newpage

% índice de anexos
%
% La idea es semejante que para  el .toc del índice, pero ahora se usa
% extensión .lof (List Of Figures) con la información de las figuras.


\ifpdf
   \pdfbookmark[1]{\'Indice de anexos}{indice de anexos}
\fi

\cabeceraEspecial{\'Indice de anexos}

\renewcommand{\listfigurename}{\'INDICE DE ANEXOS}
\listoffigures

\newpage

% Variable local para emacs, para  que encuentre el fichero maestro de
% compilación y funcionen mejor algunas teclas r�pidas de AucTeX

%%%
%%% Local Variables:
%%% mode: latex
%%% TeX-master: "../Tesis.tex"
%%% End:

\fi

% Marcamos el  comienzo de  los cap�tulos (para  la numeración  de las
% p�ginas) y ponemos la cabecera normal
\mainmatter
\restauraCabecera

\include{Capitulos/Parte1}
%---------------------------------------------------------------------
%
%                          Cap�tulo 1
%
%---------------------------------------------------------------------

\chapter{Introducci�n}

\begin{FraseCelebre}
\begin{Frase}
...
\end{Frase}
\begin{Fuente}
...
\end{Fuente}
\end{FraseCelebre}

\begin{resumen}
...
\end{resumen}


%-------------------------------------------------------------------
\section{Introducci�n}
%-------------------------------------------------------------------
\label{cap1:sec:introduccion}

...

%-------------------------------------------------------------------
\section*{\NotasBibliograficas}
%-------------------------------------------------------------------
\TocNotasBibliograficas

Citamos algo para que aparezca en la bibliograf�a\ldots
\citep{ldesc2e}

\medskip

Y tambi�n ponemos el acr�nimo \ac{CVS} para que no cruja.

Ten en cuenta que si no quieres acr�nimos (o no quieres que te falle la compilaci�n en ``release'' mientras no tengas ninguno) basta con que no definas la constante \verb+\acronimosEnRelease+ (en \texttt{config.tex}).

\begin{chart}
    \centering
    \includegraphics[width=8cm]{Imagenes/Pgfplot3d3}
    \captionof{chart}{Three dimensional graph.}
    \label{cha:chart1}
    \end{chart}
    
    \begin{figure}[h]
    \centering
    \includegraphics[width=0.5\textwidth]{Imagenes/pgfplots3dexample.png}
    \caption{Second 3D plot.}
    \label{fig:figure1}
    \end{figure}

    \begin{illustration}
        \caption{This is an example caption.}
        \label{ill:blah}  
        \begin{itemize}
          \item Possible illustration
        \end{itemize}
      \end{illustration}
%-------------------------------------------------------------------
\section*{\ProximoCapitulo}
%-------------------------------------------------------------------
\TocProximoCapitulo

...

% Variable local para emacs, para  que encuentre el fichero maestro de
% compilaci�n y funcionen mejor algunas teclas r�pidas de AucTeX
%%%
%%% Local Variables:
%%% mode: latex
%%% TeX-master: "../Tesis.tex"
%%% End:

%\include{...}
%\include{...}
%\include{...}
%\include{...}
%\include{...}

% Ap�ndices
\appendix
\include{Apendices/ParteApendices}
\include{Apendices/01ApendiceA}
%\include{...}
%\include{...}
%\include{...}

\backmatter

%
% Bibliograf�a
%

\include{Cascaras/bibliografia}

%
% �ndice de palabras
%

% S�lo  la   generamos  si  est�   declarada  \generaindice.  Consulta
% TeXiS.sty para m�s información.

% En realidad, el soporte para la generación de �ndices de palabras
% en TeXiS no est� documentada en el manual, porque no ha sido usada
% "en producción". Por tanto, el fichero que genera el �ndice
% *no* se incluye aqu� (est� comentado). Consulta la documentación
% en TeXiS_pream.tex para m�s información.
\ifx\generaindice\undefined
\else
%\include{TeXiS/TeXiS_indice}
\fi

%
% Lista de acr�nimos
%

% S�lo  lo  generamos  si  est� declarada  \generaacronimos.  Consulta
% TeXiS.sty para m�s información.


\ifx\generaacronimos\undefined
\else
\include{TeXiS/TeXiS_acron}
\fi

%
% Final
%
\include{Cascaras/fin}

\end{document}

%---------------------------------------------------------------------
%
%                          TeXiS_toc.tex
%
%---------------------------------------------------------------------
%
% TeXiS_toc.tex
% Copyright 2009 Marco Antonio Gomez-Martin, Pedro Pablo Gomez-Martin
%
% This file belongs to TeXiS, a LaTeX template for writting
% Thesis and other documents. The complete last TeXiS package can
% be obtained from http://gaia.fdi.ucm.es/projects/texis/
%
% This work may be distributed and/or modified under the
% conditions of the LaTeX Project Public License, either version 1.3
% of this license or (at your option) any later version.
% The latest version of this license is in
%   http://www.latex-project.org/lppl.txt
% and version 1.3 or later is part of all distributions of LaTeX
% version 2005/12/01 or later.
%
% This work has the LPPL maintenance status `maintained'.
% 
% The Current Maintainers of this work are Marco Antonio Gomez-Martin
% and Pedro Pablo Gomez-Martin
%
%---------------------------------------------------------------------
%
% Contiene  los  comandos  para  generar los  índices  del  documento,
% entendiendo por índices las tablas de contenidos.
%
% Genera  el  índice normal  ("tabla  de  contenidos"),  el índice  de
% figuras y el de tablas. También  crea "marcadores" en el caso de que
% se está compilando con pdflatex para que aparezcan en el PDF.
%
%---------------------------------------------------------------------

% FORMATOS DE TÍTULOS
% Capítulo
\titleformat{\chapter}
{\normalfont\large\filcenter\bfseries}{\MakeUppercase{\chaptertitlename}\ \thechapter}{1em}{}

% Seccion
\titleformat{\section}
{\normalfont\normalsize\bfseries}{\thesection}{1em}{}

% Subsección
\titleformat{\subsection}
{\normalfont\normalsize\bfseries}{\thesubsection}{1em}{}

% Agrega un punto (.) al final de las numeraciones
\renewcommand{\thechapter}{\arabic{chapter}.}
\renewcommand{\thesection}{\thechapter\arabic{section}.}
\renewcommand{\thesubsection}{\thesection\arabic{subsection}.}

% Formato del Título de los Índices de Cuadros, Figuras, Anexos
\newcommand{\formattitlelistof}[1]{%
	\thispagestyle{plain}{%
		\begin{center}\textbf{\large #1}\end{center}
	}
}

%\let\oldtituloReferenciasBibliograficas\tituloReferenciasBibliograficas
%\renewcommand{\tituloReferenciasBibliograficas}{\formattitlelistof{\oldtituloReferenciasBibliograficas}}

% Un poquito de configuración...


% Pedimos  que inserte  todos los epígrafes hasta el nivel \subsection
% en la tabla de contenidos.
\setcounter{tocdepth}{2} 

% Le  pedimos  que nos  numere  todos  los  epígrafes hasta  el  nivel
% \subsubsection en el cuerpo del documento.
\setcounter{secnumdepth}{3} 


% Nuevos tipos de listas

% Por defecto LaTeX tiene:
%    - Tabla de contenidos
%    - Índice de figuras
%    - Índice de tablas
% Lo que hace  que  si  necesitamos otros tipos de lista de cosas, por
% ejemplo:  Índice de cuadros,  Índice de ilustraciones, etc.  se debe
% definir, para eso de usa el paquete "tocloft", como dice en:
% https://texblog.org/2008/07/13/define-your-own-list-of/
%
% Ejemplo para listas de elementos de texto:
%    \newcommand{\listXname}{List of Xs}
%    \newlistof{X}{ex}{\listXname}
%    \newcommand{\X}[1]{%
%    \refstepcounter{X}
%    \par\noindent\textbf{X \theexample. #1}
%    \addcontentsline{exp}{example}
%    {\protect\numberline{\thechapter.\theexample}#1}\par}
%
% Ejemplo para listas de elementos flotantes (un elemento flotante por
% defecto son las tablas y las figuras)
%    \newfloat{project}{tbph}{lop}
%    \newcommand{\listprojectname}{List of projects}
%    \newcommand{\listofprojects}{\listof{project}{\thispagestyle{empty} \textbf{\huge \listprojectname}}}


% Índice de cuadros (List of charts)
\newfloat{chart}{tbph}{loch}[chapter]
\floatname{chart}{Cuadro}
\newcommand{\listchartname}{List of charts}
\newcommand{\listofcharts}{\listof{chart}{\formattitlelistof{\listchartname}}}

%
%
% Índice de ilustraciones (List of illustrations)
\newfloat{illustration}{tbph}{loil}[chapter]
\floatname{illustration}{Ilustraci\'on}
\newcommand{\listillustrationname}{List of illustrations}
\newcommand{\listofillustrations}{\listof{illustration}{\formattitlelistof{\listillustrationname}}}

%
%
% Índice de anexos (List of annexes)
\newcommand{\listannexename}{List of charts}
\newcommand{\listofannexes}{\vspace*{0.5in}\formattitlelistof{\listannexename}}

\newlistof{annexe}{loan}{\listofannexes}
\newcommand{\annexe}[1]{%
\refstepcounter{annexe}
\par\noindent\textbf{Annexe \theannexe. #1}
\addcontentsline{loan}{annexe}
{\protect\numberline{\theannexe}#1}\par}

% Hacemos que las numeraciones de los elementos sean consecutivas
\counterwithout{chart}{chapter}
\counterwithout{table}{chapter}
\counterwithout{figure}{chapter}
\counterwithout{illustration}{chapter}
\counterwithout{annexe}{chapter}


% Creamos los diferentes índices.

% Lo primero un  poco de trabajo en los marcadores  del PDF. No quiero
% que  salga una  entrada  por cada  índice  a nivel  0...  si no  que
% aparezca un marcador "índices", que  tenga dentro los otros tipos de
% índices.  Total, que creamos el marcador "índices".
% Antes de  la creación  de los índices,  se añaden los  marcadores de
% nivel 1.

\ifpdf
   \pdfbookmark{\'Indices}{Indices}
\fi

% Tabla de contenidos.
%
% La  inclusión  de '\tableofcontents'  significa  que  en la  primera
% pasada  de  LaTeX  se  crea   un  fichero  con  extensión  .toc  con
% información sobre la tabla de contenidos (es conceptualmente similar
% al  .bbl de  BibTeX, creo).  En la  segunda ejecución  de  LaTeX ese
% documento se utiliza para  generar la verdadera página de contenidos
% usando la  información sobre los  capítulos y demás guardadas  en el
% .toc
\ifpdf
   \pdfbookmark[1]{\'Indice general}{Indice general}
\fi

\cabeceraEspecial{\'Indice general}
\renewcommand{\contentsname}{\'INDICE GENERAL}
\tableofcontents

\newpage 


% índice de cuadros
%
% La idea es semejante que para  el .toc del índice, pero ahora se usa
% extensión .lof (List Of Figures) con la información de las figuras.
\ifpdf
   \pdfbookmark[1]{\'Indice de cuadros}{indice de cuadros}
\fi

\cabeceraEspecial{\'Indice de cuadros}

\renewcommand{\listchartname}{\'INDICE DE CUADROS}
\listofcharts

\newpage


% índice de tablas
% Como antes, pero ahora .lot (List Of Tables)
\ifpdf
   \pdfbookmark[1]{\'Indice de tablas}{indice de tablas}
\fi

\cabeceraEspecial{\'Indice de tablas}

\renewcommand{\listtablename}{\'INDICE DE TABLAS}
\listoftables

\newpage


% índice de figuras
%
% La idea es semejante que para  el .toc del índice, pero ahora se usa
% extensión .lof (List Of Figures) con la información de las figuras.
\ifpdf
   \pdfbookmark[1]{\'Indice de figuras}{indice de figuras}
\fi

\cabeceraEspecial{\'Indice de figuras}

\renewcommand{\listfigurename}{\'INDICE DE FIGURAS}
\listoffigures

\newpage


% índice de ilustraciones
%
% La idea es semejante que para  el .toc del índice, pero ahora se usa
% extensión .lof (List Of Figures) con la información de las figuras.
\ifpdf
   \pdfbookmark[1]{\'Indice de ilustraciones}{indice de ilustraciones}
\fi

\cabeceraEspecial{\'Indice de ilustraciones}

\renewcommand{\listillustrationname}{\'INDICE DE ILUSTRACIONES}
\listofillustrations

\newpage


% índice de anexos
%
% La idea es semejante que para  el .toc del índice, pero ahora se usa
% extensión .lof (List Of Figures) con la información de las figuras.
\ifpdf
   \pdfbookmark[1]{\'Indice de anexos}{indice de anexos}
\fi

\cabeceraEspecial{\'Indice de anexos}

\renewcommand{\listannexename}{\'INDICE DE ANEXOS}
\listofannexes

\newpage


% Variable local para emacs, para  que encuentre el fichero maestro de
% compilación y funcionen mejor algunas teclas rápidas de AucTeX

%%%
%%% Local Variables:
%%% mode: latex
%%% TeX-master: "../Tesis.tex"
%%% End:

%---------------------------------------------------------------------
%
%                      agradecimientos.tex
%
%---------------------------------------------------------------------
%
% agradecimientos.tex
% Copyright 2009 Marco Antonio Gomez-Martin, Pedro Pablo Gomez-Martin
%
% This file belongs to the TeXiS manual, a LaTeX template for writting
% Thesis and other documents. The complete last TeXiS package can
% be obtained from http://gaia.fdi.ucm.es/projects/texis/
%
% Although the TeXiS template itself is distributed under the 
% conditions of the LaTeX Project Public License
% (http://www.latex-project.org/lppl.txt), the manual content
% uses the CC-BY-SA license that stays that you are free:
%
%    - to share & to copy, distribute and transmit the work
%    - to remix and to adapt the work
%
% under the following conditions:
%
%    - Attribution: you must attribute the work in the manner
%      specified by the author or licensor (but not in any way that
%      suggests that they endorse you or your use of the work).
%    - Share Alike: if you alter, transform, or build upon this
%      work, you may distribute the resulting work only under the
%      same, similar or a compatible license.
%
% The complete license is available in
% http://creativecommons.org/licenses/by-sa/3.0/legalcode
%
%---------------------------------------------------------------------
%
% Contiene la p�gina de agradecimientos.
%
% Se crea como un cap�tulo sin numeraci�n.
%
%---------------------------------------------------------------------

\chapter{Agradecimientos}

\cabeceraEspecial{Agradecimientos}

\begin{FraseCelebre}
\begin{Frase}
A todos los que la presente vieren y entendieren.
\end{Frase}
\begin{Fuente}
Inicio de las Leyes Org\'anicas. Juan Carlos I
\end{Fuente}
\end{FraseCelebre}

Groucho Marx dec\'ia que encontraba a la televisi\'on muy educativa porque
cada vez que alguien la encend\'ia, \'el se iba a otra habitaci\'on a leer
un libro. Utilizando un esquema similar, nosotros queremos agradecer
al Word de Microsoft el habernos forzado a utilizar \LaTeX. Cualquiera
que haya intentado escribir un documento de m\'as de 150 p\'aginas con
esta aplicaci\'on entender\'a a qu\'e nos referimos. Y lo decimos porque
nuestra andadura con \LaTeX\ comenz\'o, precisamente, despu\'es de
escribir un documento de algo m\'as de 200 p\'aginas. Una vez terminado
decidimos que nunca m\'as pasar\'iamos por ah\'i. Y entonces ca\'imos en
\LaTeX.

Es muy posible que hubieramos llegado al mismo sitio de todas formas,
ya que en el mundo acad\'emico a la hora de escribir art\'iculos y
contribuciones a congresos lo m\'as extendido es \LaTeX. Sin embargo,
tambi\'en es cierto que cuando intentas escribir un documento grande
en \LaTeX\ por tu cuenta y riesgo sin un enlace del tipo ``\emph{Author
  instructions}'', se hace cuesta arriba, pues uno no sabe por donde
empezar.

Y ah\'i es donde debemos agradecer tanto a Pablo Gerv\'as como a Miguel
Palomino su ayuda. El primero nos ofreci\'o el c\'odigo fuente de una
programaci\'on docente que hab\'ia hecho unos a\~{n}os atr\'as y que nos sirvi\'o
de inspiraci\'on (por ejemplo, el fichero \texttt{guionado.tex} de
\texis\ tiene una estructura casi exacta a la suya e incluso puede
que el nombre sea el mismo). El segundo nos dej\'o husmear en el c\'odigo
fuente de su propia tesis donde, adem\'as de otras cosas m\'as
interesantes pero menos curiosas, descubrimos que a\'un hay gente que
escribe los acentos espa\~{n}oles con el \verb+\'{\i}+.

No podemos tampoco olvidar a los numerosos autores de los libros y
tutoriales de \LaTeX\ que no s\'olo permiten descargar esos manuales sin
coste adicional, sino que tambi\'en dejan disponible el c\'odigo fuente.
Estamos pensando en Tobias Oetiker, Hubert Partl, Irene Hyna y
Elisabeth Schlegl, autores del famoso ``The Not So Short Introduction
to \LaTeXe'' y en Tom\'as Bautista, autor de la traducci\'on al espa\~{n}ol. De
ellos es, entre otras muchas cosas, el entorno \texttt{example}
utilizado en algunos momentos en este manual.

Tambi\'en estamos en deuda con Joaqu\'in Ataz L\'opez, autor del libro
``Creaci�n de ficheros \LaTeX\ con {GNU} Emacs''. Gracias a �l dejamos
de lado a WinEdt y a Kile, los editores que por entonces utiliz�bamos
en entornos Windows y Linux respectivamente, y nos pasamos a emacs. El
tiempo de escritura que nos ahorramos por no mover las manos del
teclado para desplazar el cursor o por no tener que escribir
\verb+\emph+ una y otra vez se lo debemos a �l; nuestro ocio y vida
social se lo agradecen.

Por �ltimo, gracias a toda esa gente creadora de manuales, tutoriales,
documentaci�n de paquetes o respuestas en foros que hemos utilizado y
seguiremos utilizando en nuestro quehacer como usuarios de
\LaTeX. Sab�is un mont�n.

Y para terminar, a Donal Knuth, Leslie Lamport y todos los que hacen y
han hecho posible que hoy puedas estar leyendo estas l�neas.

\endinput
% Variable local para emacs, para  que encuentre el fichero maestro de
% compilaci�n y funcionen mejor algunas teclas r�pidas de AucTeX
%%%
%%% Local Variables:
%%% mode: latex
%%% TeX-master: "../Tesis.tex"
%%% End:

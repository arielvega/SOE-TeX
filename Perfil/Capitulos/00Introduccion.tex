%---------------------------------------------------------------------
%
%                           Introducción
%
%---------------------------------------------------------------------
% Guardamos el comando antes de redefinirlo
\let\oldthesection\thesection
% Cambiamos el formato de la numeración de la subseccion de (0.1, 0.2, 0.3, etc) a (1, 2, 3, etc)
\renewcommand*{\thesection}{\arabic{section}.}

% Crea un capítulo sin numeración
\chapter*{INTRODUCCI\'ON}
% Agrega un capítulo ficticio a la Tabla de Contenidos (ToC)
\addcontentsline{toc}{chapter}{INTRODUCCI\'ON}  


\textbf{(M\'inimo 1 p\'agina y hasta 2 p\'aginas)} Su prop\'osito es brindar informaci\'on al lector 
acerca del tema de modo que permita la comprensi\'on y evaluaci\'on de resultados del propio trabajo
(``enamorar al que lee'' sobre el tema de investigaci\'on) en un momento posterior. Es decir, que la 
introducci\'on es la fundamentaci\'on del proyecto en forma resumida. En ella se deben exponer 
brevemente, pero con absoluta claridad, la novedad y actualidad del tema. El \'ultimo p\'arrafo es 
la presentaci\'on del trabajo que se va a realizar. \hl{REDACCI\'ON EN TERCERA PERSONA DEL SINGULAR}



%-------------------------------------------------------------------
\section{Antecedentes del problema}
%-------------------------------------------------------------------
\label{cap0:sec:antecedentes_del_problema}

\textbf{(Hasta 2 p\'{a}ginas)} La informaci\'{o}n que se detalla en este punto debe considerar una 
breve descripci\'{o}n del contexto que rodea al problema, situaci\'{o}n problem\'{a}tica en general, 
todo hecho anterior a la concepci\'{o}n del trabajo de investigaci\'{o}n, as\'{\i} como, las referencias 
a circunstancias relacionadas con el mismo, motivos que llevan al investigador a realizar el trabajo 
(vac\'{\i}o del conocimiento), es una forma de aclarar, juzgar e interpretar la investigaci\'{o}n 
planteada. Se debe mencionar investigaciones anteriores (preferentemente art\'{\i}culos 
cient\'{\i}ficos) o si existen proyectos ejecutados o en ejecuci\'{o}n que pretendan resolver el 
mismo problema y los resultados alcanzados.

Las referencias bibliogr\'{a}ficas deben ser gestionadas con el Gestor de Word u otro como Mendeley, 
Zotero. 



%-------------------------------------------------------------------
\section{Formulaci\'on del problema}
%-------------------------------------------------------------------
\label{cap0:sec:formulacion_del_problema}

\textbf{(3 a 5 l\'ineas)} El Problema es aquello que se ignora acerca del tema de inter\'{e}s y que 
requiere ser aclarado procediendo sistem\'{a}ticamente, por lo que, a mayor exactitud en su 
formulaci\'{o}n, mayor probabilidad de lograr respuestas satisfactorias. Puede ser formulado en 
forma de preguntas o afirmaci\'{o}n como relaci\'{o}n entre dos o m\'{a}s conceptos o variables, 
posibilitando la prueba emp\'{\i}rica de las mismas. En ambos casos, debe especificar la 
poblaci\'{o}n o sujetos de la investigaci\'{o}n y el contexto donde se investigar\'{a}.


%-------------------------------------------------------------------
\subsection{Objeto de estudio}
%-------------------------------------------------------------------
\label{cap0:sub:objecto_de_estudio}

\textbf{(1 a 2 l\'{\i}neas)}, es la parte de la realidad objetiva donde se manifiesta la necesidad 
o el problema de investigaci\'{o}n y sobre la cual act\'{u}a el sujeto pr\'{a}ctica o 
te\'{o}ricamente en busca de la soluci\'{o}n adecuada. Responde al \textbf{``QU\'{E} VOY A ESTUDIAR''}


%-------------------------------------------------------------------
\subsection{Campo de acci\'on}
%-------------------------------------------------------------------
\label{cap0:sub:campo_de_accion}

\textbf{(1 a 2 l\'{\i}neas)}, es la parte m\'{a}s concreta de la realidad donde se manifiesta la 
necesidad o el problema a ser resuelto en el proceso de investigaci\'{o}n. Es un concepto m\'{a}s 
reducido que el objeto de estudio (realidad objetiva), siendo a la vez parte componente del mismo 
y es justamente donde se articulan los objetivos. Responde a 
\textbf{``QU\'{E} SE TRANSFORMA DURANTE LA INVESTIGACI\'{O}N''}



%-------------------------------------------------------------------
\section{Objetivos de la investigaci\'on}
%-------------------------------------------------------------------
\label{cap0:sec:objetivos_de_la_investigacion}

%-------------------------------------------------------------------
\subsection{Objetivo general}
%-------------------------------------------------------------------
\label{cap0:sub:objetivo_general}

Es la formulaci\'{o}n coherente de la soluci\'{o}n propuesta por el investigador a la necesidad o 
problema de investigaci\'{o}n planteado. Debe reflejar la esencia del planteamiento del problema y 
la idea expresada en el t\'{\i}tulo del proyecto de investigaci\'{o}n. Su caracter\'{\i}stica 
b\'{a}sica es que hace referencia clara a la realizaci\'{o}n de lo propuesto \textbf{``QU\'{E}''} y 
\textbf{``PARA QU\'{E}''}, por lo tanto, debe ser medible, incluir el \textbf{``C\'{O}MO''} se va 
a realizar el ``qu\'{e}'', finalmente \textbf{``D\'{O}NDE''}, que se refiere al contexto de 
investigaci\'{o}n. 


%-------------------------------------------------------------------
\subsection{Objetivos espec\'ificos}
%-------------------------------------------------------------------
\label{cap0:sub:objetivos_especificos}

Resultan de la operativizaci\'{o}n m\'{a}s concreta del Objetivo General y buscan su realizaci\'{o}n 
es el c\'{o}mo de la investigaci\'{o}n; c\'{o}mo lograr el objetivo general (pasos m\'{a}s 
peque\~{n}os), son un anticipo del dise\~{n}o de la investigaci\'{o}n, responden a la estructura
\textbf{``QU\'{E}''} y \textbf{``PARA QU\'{E}''} e igualmente deben ser medibles. 
\hl{\textbf{SE DEBEN NUMERAR CONSECUTIVAMENTE, ALINEADOS A LA IZQUIERDA}}



%-------------------------------------------------------------------
\section{Justificaci\'on de la investigaci\'on }
%-------------------------------------------------------------------
\label{cap0:sec:justificacion_de_la_investigacion}

\textbf{(Hasta 1 p\'{a}gina)}, sustentar con argumentos convincentes la realizaci\'{o}n de la investigaci\'{o}n.
Puede ayudar, responder las interrogantes: \textquestiondown{}Es de actualidad el tema? \textquestiondown{}Qu\'{e}
acciones se han realizado al respecto?, \textquestiondown{}Se agravar\'{a} con el transcurso del tiempo?,
\textquestiondown{}Es viable?, \textquestiondown{}Es factible?, \textquestiondown{}Es de relevancia social?,
\textquestiondown{}Qu\'{e} efectos negativos se desencadenan como resultado del estado actual?,
\textquestiondown{}Qu\'{e} efectos positivos se tendr\'{\i}an si se realiza la investigaci\'{o}n?
De acuerdo con el criterio del investigador, se puede distinguir en econ\'{o}mica, social, te\'{o}rica, pr\'{a}ctica,
metodol\'{o}gica, entre otras. 


Se recomienda separar en diferentes tipos de justificaci\'{o}n: metodol\'{o}gica, econ\'{o}mica, 
social, pr\'{a}ctica, te\'{o}rica. 


\hl{SE DEBEN ABORDAR AL MENOS TRES ASPECTOS} 


%-------------------------------------------------------------------
\subsection{Justificaci\'{o}n pr\'{a}ctica}
%-------------------------------------------------------------------
\label{cap0:sub:justificacion_practica}


%-------------------------------------------------------------------
\subsection{Justificaci\'{o}n metodol\'{o}gica}
%-------------------------------------------------------------------
\label{cap0:sub:justificacion_metodologica}


%-------------------------------------------------------------------
\subsection{Justificaci\'{o}n econ\'{o}mica}
%-------------------------------------------------------------------
\label{cap0:sub:justificacion_economica}


%-------------------------------------------------------------------
\section{Formulaci\'on de la construcci\'on te\'orica. Hip\'otesis para Defender.}
%-------------------------------------------------------------------
\label{cap0:sec:formulacion_de_la_construccion_teorica}

Es una construcci\'{o}n te\'{o}rica que plantea una posible descripci\'{o}n, comprensi\'{o}n, 
correlaci\'{o}n o explicaci\'{o}n como respuesta al problema de estudio. Se trata de una 
relaci\'{o}n entre \textbf{dos o m\'{a}s variables} expresadas como hechos, fen\'{o}menos, 
factores o entidades y que debe ser sometida a prueba para ser aceptada como v\'{a}lida. Debe 
responder al problema de investigaci\'{o}n.

\hl{ESTA EXPLICACI\'{O}N DEBE SER ELIMINADA} 


%-------------------------------------------------------------------
\subsection{Identificaci\'on de las variables}
%-------------------------------------------------------------------
\label{cap0:sub:identificacion_de_las_variables}

%-------------------------------------------------------------------
\subsubsection{Variable independiente}
%-------------------------------------------------------------------
\label{cap0:sub:variable_independiente}

%-------------------------------------------------------------------
\subsubsection{Variable dependiente}
%-------------------------------------------------------------------
\label{cap0:sub:variable_dependiente}



% Se restablece el formato de la numeración de la subseccion
\let\thesection\oldthesection


% Variable local para emacs, para  que encuentre el fichero maestro de
% compilación y funcionen mejor algunas teclas rápidas de AucTeX
%%%
%%% Local Variables:
%%% mode: latex
%%% TeX-master: "../Tesis.tex"
%%% End:

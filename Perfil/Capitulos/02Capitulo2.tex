%---------------------------------------------------------------------
%
%                          Capítulo 2
%
%---------------------------------------------------------------------

\chapter{DISE\~{N}O METODOL\'{O}GICO}


%-------------------------------------------------------------------
\section{Tipo, Enfoque y Alcance de la Investigaci\'on }
%-------------------------------------------------------------------
\label{cap2:sec:tipo_enfoque_y_alcance_de_la_investigacion}


%-------------------------------------------------------------------
\section{Delimitaci\'on de la Investigaci\'on}
%-------------------------------------------------------------------
\label{cap2:sec:delimitacion_de_la_investigacion}

\textbf{Delimitaci\'{o}n tem\'{a}tica:} Especifica el l\'{\i}mite del estudio en cuanto a la aplicaci\'{o}n 
de teor\'{\i}as indicando en algunos casos qu\'{e} teor\'{\i}as o \textbf{AUTORES} se tomar\'{a}n en cuenta, 
\'{a}reas del conocimiento para la ciencia en cuesti\'{o}n, as\'{\i} tambi\'{e}n normas, 
regulaciones, decretos u otros. 

\textbf{Delimitaci\'{o}n espacial:} Se debe especificar el lugar geogr\'{a}fico donde se realiza el estudio 
actual y por otra parte el lugar geogr\'{a}fico donde se aplicar\'{a}n los resultados en caso de 
ser diferentes.

\textbf{Delimitaci\'{o}n Temporal:} Se debe definir el periodo de tiempo en que se realizar\'{a} el trabajo 
de investigaci\'{o}n incluyendo fechas concretas estimadas. Por otra parte, se debe tambi\'{e}n 
indicar el l\'{\i}mite temporal de la validez de la propuesta de soluci\'{o}n o cu\'{a}ndo se debe 
realizar una actualizaci\'{o}n de esta (bajo qu\'{e} condiciones).


%-------------------------------------------------------------------
\section{Definici\'{o}n Conceptual de las Variables}
%-------------------------------------------------------------------
\label{cap2:sec:definicion_conceptual_de_las_variables}

Aqu\'i se deben definir las variables (independiente y dependiente) a partir de la sistematizaci\'on 
realizada a los diferentes autores (textual) o por elaboraci\'{o}n propia. \hl{ESTA EXPLICACI\'{O}N 
DEBE SER ELIMINADA}


%-------------------------------------------------------------------
\section{Definici\'{o}n Operacional de las Variables}
%-------------------------------------------------------------------
\label{cap2:sec:definicion_operacional_de_las_variables}

La operacionalizaci\'{o}n de las variables en dimensiones e indicadores seg\'{u}n la definici\'{o}n 
en el ep\'{\i}grafe anterior, se debe referenciar el cuadro que sigue a continuaci\'{o}n. \hl{ESTA 
EXPLICACI\'{O}N DEBE SER ELIMINADA}


\begin{chart}
    \begin{threeparttable}
        \caption{Operacionalizaci\'{o}n de las variables}
        \label{cap2:cha:operacionalizacion_de_las_variables}
        \begin{tabulary}{\textwidth}{|p{0.3\linewidth}|p{0.3\linewidth}|p{0.3\linewidth}|}
            \hline
                \rowcolor{lightgray}\thead{VARIABLE} &
                \thead{DIMENSIONES} &
                \thead{INDICADORES} \\
            \hline
                \multirow{2}{*}{1. Var independiente} & {1.1.} & {1.1.1.\newline1.1.2.} \\
                \cline{2-3}
                & {1.2.} & {1.2.1.\newline1.2.2.\newline1.2.3.} \\
            \hline
                \multirow{2}{*}{2. Var dependiente} & {2.1.} & {2.1.1.\newline2.1.2.\newline2.1.3.} \\
                \cline{2-3}
                & {2.2.} & {2.2.1.\newline2.2.2.} \\
                \cline{2-3}
                & {2.3.} & {2.3.1.\newline2.3.2.\newline2.3.3.} \\
            \hline
        \end{tabulary}
        \begin{tablenotes}
            \small\item \textbf{Fuente:} Elaboraci\'{o}n propia.
        \end{tablenotes}
    \end{threeparttable}
\end{chart}

%-------------------------------------------------------------------
\section{M\'{e}todos de Investigaci\'{o}n}
%-------------------------------------------------------------------
\label{cap2:sec:metodos_de_investigacion}

Se seleccionan los m\'{e}todos que se van a aplicar de acuerdo con la etapa de la investigaci\'{o}n, 
en forma de \textit{check list}, se explica para qu\'{e} servir\'{a} o c\'{o}mo se utilizar\'{a} en la etapa 
espec\'{\i}fica. \hl{ESTA EXPLICACI\'{O}N DEBE SER ELIMINADA} 

\begin{itemize}
    \item M\'{e}todo de \'{I}ndice  

    \item M\'{e}todo de Mapeo 

    \item M\'{e}todo de An\'{a}lisis documental

    \item M\'{e}todo de Enfoque de Sistema 

    \item M\'{e}todo de Modelaci\'{o}n 

    \item M\'{e}todo de Observaci\'{o}n 

    \item Experimental, cuasi experimental o preexperimental 
\end{itemize}

%-------------------------------------------------------------------
\section{T\'ecnicas de Recolecci\'on de Datos de la Investigaci\'on }
%-------------------------------------------------------------------
\label{cap2:sec:tecnicas_de_recoleccion_de_datos_de_la_investigacion }

\begin{itemize}
    \item Aqu\'{\i} se identifica \textbf{QU\'{E}} t\'{e}cnica se va a utilizar, \textbf{A QUI\'{E}N} se le 
          va a aplicar y se explica \textbf{PARA QU\'{E}} se utilizar\'{a} (en forma de \textit{check list})  
\end{itemize}


%-------------------------------------------------------------------
\section{Instrumentos de Investigaci\'{o}n}
%-------------------------------------------------------------------
\label{cap2:sec:instrumentos_de_investigacion}

\begin{itemize}
    \item Se mencionan de acuerdo con los m\'{e}todos y t\'{e}cnicas que se identifican anteriormente (en 
forma de \textit{check list}) 
\end{itemize}

%-------------------------------------------------------------------
\section{Poblaci\'{o}n y Muestra}
%-------------------------------------------------------------------
\label{cap2:sec:poblacion_y_muestra}

Aqu\'{i} se identifica la poblaci\'{o}n y la muestra para cada uno de los instrumentos de las 
t\'{e}cnicas si no fueran la misma, o se expone de manera general, en caso de coincidir.


%-------------------------------------------------------------------
\section{An\'{a}lisis de los Datos}
%-------------------------------------------------------------------
\label{cap2:sec:analisis_de_los_datos}

Aqu\'{i} se debe escribir uno o dos p\'{a}rrafos que explique c\'{o}mo ser\'{a}n procesados los
datos obtenidos de la aplicaci\'{o}n de los instrumentos de investigaci\'{o}n en el proceso de 
diagn\'{o}stico y la validaci\'{o}n de la propuesta de soluci\'{o}n al problema de 
investigaci\'{o}n. \hl{ESTA EXPLICACI\'{O}N DEBE SER ELIMINADA}


%-------------------------------------------------------------------
\section{Cronograma de Investigaci\'{o}n}
%-------------------------------------------------------------------
\label{cap2:sec:cronograma_de_investigacion}

Se debe exponer mediante diagrama de Grant la planificaci\'{o}n, en tiempo, de la investigaci\'{o}n,
 referenciando el cuadro que aparece a continuaci\'{o}n \hl{ESTA EXPLICACI\'{O}N DEBE SER ELIMINADA}


 \begin{chart}
    \begin{threeparttable}
        \caption{Planificaci\'{o}n de la investigaci\'{o}n}
        \label{cap2:cha:planificacion_de_la_investigacion}
        \resizebox{\textwidth}{!}{
        \begin{tabulary}{1.0\textwidth}{|c|c|c|c|c|c|c|c|c|c|c|c|c|c|c|c|c|c|c|c|c|}
            \hline
                \rowcolor{lightgray}
                {}&
                \multicolumn{20}{|c|}{\textbf{2018}}\\
                \cline{2-21}
                \rowcolor{lightgray}
                \multirow{-2}{*}{\textbf{ACTIVIDADES}} &
                \multicolumn{4}{|c|}{\textbf{ABRIL}} &
                \multicolumn{4}{|c|}{\textbf{MAYO}} &
                \multicolumn{4}{|c|}{\textbf{JUNIO}} &
                \multicolumn{4}{|c|}{\textbf{JULIO}} &
                \multicolumn{4}{|c|}{\textbf{AGOSTO}} \\
            \hline
            \multicolumn{1}{|l|}{Elaboraci\'{o}n del perfil de tesis}
                 & {} & {} & {x} & {x} 
                 & {x} & {x} & {} & {} 
                 & {} & {} & {} & {} 
                 & {} & {} & {} & {} 
                 & {} & {} & {} & {} \\
            \hline
            \multicolumn{1}{|l|}{Presentaci\'{o}n del perfil de tesis}
                 & {} & {} & {} & {} 
                 & {} & {} & {x} & {} 
                 & {} & {} & {} & {} 
                 & {} & {} & {} & {} 
                 & {} & {} & {} & {} \\
            \hline
            \multicolumn{1}{|l|}{Revisi\'{o}n por el tribunal}
                 & {} & {} & {} & {} 
                 & {} & {} & {} & {x} 
                 & {} & {} & {} & {} 
                 & {} & {} & {} & {} 
                 & {} & {} & {} & {} \\
            \hline
            \multicolumn{1}{|l|}{Defensa ante el tribunal}
                 & {} & {} & {} & {} 
                 & {} & {} & {} & {} 
                 & {x} & {} & {} & {} 
                 & {} & {} & {} & {} 
                 & {} & {} & {} & {} \\
            \hline
            \multicolumn{1}{|l|}{Elaboraci\'{o}n del marco te\'{o}rico}
                 & {} & {} & {} & {} 
                 & {} & {} & {} & {} 
                 & {x} & {x} & {x} & {x} 
                 & {} & {} & {} & {} 
                 & {} & {} & {} & {} \\
            \hline
            \multicolumn{1}{|l|}{Diagn\'{o}stico del estado actual}
                 & {} & {} & {} & {} 
                 & {} & {} & {} & {} 
                 & {} & {} & {} & {} 
                 & {x} & {x} & {x} & {x} 
                 & {} & {} & {} & {} \\
            \hline
            \multicolumn{1}{|l|}{Propuesta de soluci\'{o}n}
                 & {} & {} & {} & {} 
                 & {} & {} & {} & {} 
                 & {} & {} & {} & {} 
                 & {} & {} & {} & {} 
                 & {x} & {x} & {x} & {x} \\
            \hline
            \multicolumn{1}{|l|}{Presentaci\'{o}n del borrador inicial}
                 & {} & {} & {} & {} 
                 & {} & {} & {} & {} 
                 & {} & {} & {} & {} 
                 & {} & {} & {} & {} 
                 & {} & {} & {} & {} \\
            \hline
            \multicolumn{1}{|l|}{Revisi\'{o}n por el tribunal}
                 & {} & {} & {} & {} 
                 & {} & {} & {} & {} 
                 & {} & {} & {} & {} 
                 & {} & {} & {} & {} 
                 & {} & {} & {} & {} \\
            \hline
            \multicolumn{1}{|l|}{Defensa ante el tribunal}
                 & {} & {} & {} & {} 
                 & {} & {} & {} & {} 
                 & {} & {} & {} & {} 
                 & {} & {} & {} & {} 
                 & {} & {} & {} & {} \\
            \hline
            \multicolumn{1}{|l|}{Correcciones finales}
                 & {} & {} & {} & {} 
                 & {} & {} & {} & {} 
                 & {} & {} & {} & {} 
                 & {} & {} & {} & {} 
                 & {} & {} & {} & {} \\
            \hline
            \multicolumn{1}{|l|}{Defensa final de tesis}
                 & {} & {} & {} & {} 
                 & {} & {} & {} & {} 
                 & {} & {} & {} & {} 
                 & {} & {} & {} & {} 
                 & {} & {} & {} & {} \\
            \hline
        \end{tabulary}}
        \begin{tablenotes}
            \small\item \textbf{Fuente:} Elaboraci\'{o}n propia, 2022.
        \end{tablenotes}
    \end{threeparttable}
\end{chart}

% Variable local para emacs, para  que encuentre el fichero maestro de
% compilación y funcionen mejor algunas teclas rápidas de AucTeX
%%%
%%% Local Variables:
%%% mode: latex
%%% TeX-master: "../Tesis.tex"
%%% End:

%---------------------------------------------------------------------
%
%                          Parte 1
%
%---------------------------------------------------------------------
%
% Parte1.tex
% Copyright 2009 Marco Antonio Gomez-Martin, Pedro Pablo Gomez-Martin
%
% This file belongs to the TeXiS manual, a LaTeX template for writting
% Thesis and other documents. The complete last TeXiS package can
% be obtained from http://gaia.fdi.ucm.es/projects/texis/
%
% Although the TeXiS template itself is distributed under the 
% conditions of the LaTeX Project Public License
% (http://www.latex-project.org/lppl.txt), the manual content
% uses the CC-BY-SA license that stays that you are free:
%
%    - to share & to copy, distribute and transmit the work
%    - to remix and to adapt the work
%
% under the following conditions:
%
%    - Attribution: you must attribute the work in the manner
%      specified by the author or licensor (but not in any way that
%      suggests that they endorse you or your use of the work).
%    - Share Alike: if you alter, transform, or build upon this
%      work, you may distribute the resulting work only under the
%      same, similar or a compatible license.
%
% The complete license is available in
% http://creativecommons.org/licenses/by-sa/3.0/legalcode
%
%---------------------------------------------------------------------

% Definición de la primera parte del manual

\partTitle{Conceptos b\'asicos}

\partDesc{Esta primera parte del manual presenta los conceptos b\'asicos
  de \texis. Contiene un cap\'itulo de introducci\'on, seguido de una
  descripci\'on de la estructura de \texis\ y c\'omo se genera el
  documento final, para terminar con un cap\'itulo en el que se describe
  el proceso de edici\'on sugerido y los comandos que \texis\
  proporciona para facilitar dicho proceso.}

\partBackText{En realidad la divisi\'on por partes del manual no aporta
  demasiado al lector; se ha dividido en varias partes debido a que,
  en la pr\'actica, el c\'odigo de este manual sirve como ejemplo de uso
  de \texis.

  En un contexto distinto, es posible que un manual de este tipo no
  habr\'ia tenido estas partes as\'i de diferenciadas.}

\makepart
